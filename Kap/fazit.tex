\chapter{Fazit}

Bei diesem Versuch konnte wertvolles Wissen über das Rasterkraftmikroskop erlangt werden.
Mit teils bekannter und teils unbekannter Theorie konnte nun das Prinzip einer bildlichen Darstellung in fast atomarer Auflösung verstanden werden.
Es wurden verschiedene Topographien von einer Kalibrierungsprobe aufgenommen, um damit zu testen, ob das Mikroskop richtig kalibriert war.
Schön zu sehen war, wie sich die Abbildung durch die Vorwärts-, bzw. Rückwärtsbewegung ändert.
Auch wurde eine Resonanzkurve des Cantilevers aufgenommen, um die Resonanzfrequenz zu erhalten.
Unsere Messwerte stimmten sehr exakt mit denen des Herstellers überein, minimale Fehler sind mit diesem Aufbau, wegen Luftverwirbelungen und Erschütterungen, unvermeidbar.
Danach wurde die Topographie einer CD ausgelesen, um deren Kapazität berechnen zu können.
Das berechnete Ergebnis stimmt, in großer Näherung, mit der tatsächlichen Kapazität von Handelsüblichen CD-Roms überein.
