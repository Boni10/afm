\chapter{Einleitung}

Die Familie der Rastersondenmikroskope wurde im Jahre 1982 mit der Erfindung des Rastertunnelmikroskops (scanning tunneling microscope, STM) ”gegründet“.
Das STM erlaubte erstmals die Abbildung von Oberflächen im Ortsraum mit atomarer Auflösung.
Der Abbildungsprozess im STM beruht auf dem Tunnelstrom zwischen einer scharfen Metallspitze und der Probenoberfläche. 
Dieser Strom zeigt eine exponentielle Abstandsabhängigkeit. 
Wenn man die Spitze mit ausreichender Präzision (im Angstroem-Bereich) über der Probe positionieren kann und dabei den Tunnelstrom misst, kann man durch Abrastern von Linien konstanten Tunnelstroms ein Bild der Probenoberfläche gewinnen. 
Das STM hat zwareine hohe räumliche Auflösung, ist aber auf leitende Proben beschränkt.

Das Rasterkraftmikroskop (atomic force microscope AFM, oder scanning force microscope SFM), das 1986 erstmals vorgestellt wurde, umgeht diese Einschränkung, indem statt des Tunnelstroms die Kräfte zwischen Spitze und Probe gemessen werden. 
Die Spitze wird an einer Balkenfeder (”Cantilever“) angebracht, und auf die Spitze wirkende Kräfte führen im einfachsten Fall zu einer Verbiegung der Feder. Diese Verbiegung lässt sich mit verschiedenen  Verfahren  messen  und  als  Regelsignal  nutzen.
Rastert  man  die  Probe in  der x-y-Ebene  und  regelt  man z auf  konstante  Verbiegung  der  Feder, so erhält man eine Karte konstanter Kraft, die als Oberflächentopographie interpretiert werden kann.
Die Einsatzbereiche der Kraftmikroskopie sind vielfältig und reichen von Abbildung  und  Manipulation  atomarer  Strukturen über  Charakterisierung von Kristallwachstum zur Untersuchung von biologischen Systemen und Qualitätssicherung und Fehleranalyse bei Festplatten.

(Quelle: \url{http://www.physik.uni-regensburg.de/studium/praktika/f/fpAFM2010.pdf})
